\documentclass[12pt]{article}

\usepackage[margin=1in]{geometry} 
\usepackage{amsmath, amssymb}    
\usepackage{graphicx}            
\usepackage{booktabs}            
\usepackage{setspace}            
\usepackage{natbib}             

\onehalfspacing                 

\begin{document}

\title{Money, Credit, and Macroeconomic Analysis of Ghana}
\author{Your Name}
\date{\today}                    
\maketitle

\begin{abstract}
This paper studies the relationships between money growth, inflation,
credit, and macro-financial variables in Ghana.
\end{abstract}

\section{Introduction}

Historically, Ghana has struggled with repeated bouts of high inflation and currency depreciation (especially mid-2010s and 2022-23), which have major 
implications for the economic lives of its citizens. Ghana has also faced rising public debt and pressures on FX reserves, which
have been supplemented with IMF program disbursements over the last ~50 years in attempts to calm the issue. 

The goal of this paper is to analyze Ghana's monthly macro and financial data in order to study the relationships between monetary policy, money/credit growth, FX pressures, and IMF 
programs. This will allow us to better understand how the Bank of Ghana (BoG) systematically reacts to inflation, how strongly money and credit growth are linked to inflation and activity, and
also how reserves adn FX depreciation behave around IMF programs. 

Before analyzing the data, we can infer based off other instances of macroeconomic conditions that the BoG's policy will react to inflation in an imperfect mannor. We
can also infer that money and credit growth will not be fully indicative of inflation or growth. Lastly, we can infer that major FX crises, such as big 12-month depreciations, result
in (or line up closely with) IMF arrangements.

\section{Data and Methodology}

Initial data sources for this research included the BoG's macro sheet, which included features such as broad money (M2+), private credit, interest rates, CIEA growth, reserves, and NPLs.
To access GDP and unemployment data, we sourced quarterly and annual series, which we transformed to yoy and matched to monthly. FX data was taken from interbank USD/GHS daily, which was 
engineered into monthly averages, which was then used to build a 12-month depreciation rate series. IMF arrangements were pulled using their start dates and agreed amounts.

In terms of feature engineering, we built year-on-year averages to build the M2+ year-over-year, private credit year-over-year, real private credit year-over-year, and gdp (non-oil) year-over-year.
We also built real policy rate, real interbank rate, and real lending rate by subtracting the inflation rate from the nominal. In regards to FX, we calculated the twelve-month percent change in USD/GHS.
A twelve-month net change in reserves was also engineered, along with lags in inflation, and real credit.

Tools consisted of feature correlation matrices, OLS regressions with a single key regressor. We also used ADF tests to check for stationarity and avoid using non-stationary features.


\section{Monetary Policy, Inflation, and Money Growth}

\subsection{Inflation vs Policy Rate}

\begin{figure}[htbp]
  \centering
  \includegraphics[width=0.8\textwidth]{../figures/analysis/inflationPolicyRateRealPolicyRate.png}
  \caption{Time series - headline inflation, policy rate, real policy rate.}
  \label{fig:inflationvspolicy}
\end{figure}

To start off the analysis, we compared inflation with policy and real policy rate. As you can see in the visual, inflation has two major peaks (~20 around mid-2010s and 50 in 2022-23).
The policy rate moves with inflation but lags it, while the real policy rate naturally turns sharply negative in high-inflation episodes. Clearly, monetary policy often falls behind the curve.
When real rates become deeply negative, inflation can be entrenched by expansionary policy.

\subsection{Simple Policy Reaction Function}

\begin{figure}[htbp]
  \centering
  \includegraphics[width=0.8\textwidth]{../figures/analysis/ChangePolicyRateVsLaggedInflationScatter.png}
  \caption{Change in policy rate vs lagged inflation}
  \label{fig:policyvslaggedinflation}
\end{figure}

When running a regression on change in policy rate and lagged inflation we found a slope of 0.04, indicating that a one percentage point increase in last months inflation results in a 0.04 percentage point change in policy rate this month.
The p-value of this regression was <0.001, indicating statistical significance, with a $R^2$ of around 0.13 indicating only ~13 percent of policy moves are explained by lagged inflation.

This can be interpreted as the fact that the BoG does not systematically respond to inflation, but rather responds modestly. Large policy moves likely reflect other factors such as FX pressure. This matches the first visual, which indicated that 
policy follows inflation but not aggressively enough to keep real rates positive.

\subsection{Money Growth and Inflation}





\section{Conclusion}

\bibliographystyle{apalike}      
\bibliography{references}        

\end{document}

