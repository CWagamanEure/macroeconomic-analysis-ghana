\documentclass[12pt]{article}

\usepackage[margin=1in]{geometry} 
\usepackage{amsmath, amssymb}    
\usepackage{graphicx}            
\usepackage{booktabs}            
\usepackage{setspace}            
\usepackage{natbib}             

\onehalfspacing                 

\begin{document}

\title{Money, Credit, and Macroeconomic Analysis of Ghana}
\author{Cory Wagaman-Eure}
\date{\today}                    
\maketitle

\begin{abstract}
This paper studies the relationships between money growth, inflation,
credit, and macro-financial variables in Ghana.
\end{abstract}

\section{Introduction}

This paper looks at how Ghana's macroeconomic conditions have evolved since the mid-2000s, with a focus on inflation, credit growth, and pressure on the exchange rate. It will also cover the current macroeconomic state of Ghana and policy recomendations. Over
the period of the mid-2000s to present Ghana has gone through several boom-bust cycles, large swings in inflation, and repeated IMF programs. This paper will look at a few key graphs and regressions to determine how key variables move together.

The main questions I ask are: How closely does Ghana's inflation track money growth, and how aggressively does the Bank of Ghana (BoG) move its policy rate in response to inflation? By answering this question, we can gage better responses for improvement in 
economic conditions. We also ask how strongly growth in private credit is linked to real economic activity and to bank health. Last, we ask how exchange rate depreciation and international reserves behave around IMF programs.

To answer these questions, I put together monthly data from the BoG on inflation, policy and lending rates, monetary aggregates, private sector credit, the Composite Index of Economic Activity, and banking sector indicators. I combine these with
interbank exchange rate data and a list of Ghana's IMF arrangements. I then construct year-on-year growth rates and run OLS regressions to determine patterns in Ghana's macro history.

\section{Data and Methodology}

The main data source for this project is the BoG's published macroeconomic time series. From there I take monthly series for narrow money (M1), broad money (M2), the policy rate, interbank and lending rates, the Ghana Reference rate, headline and component inflation,
private sector credit, the CIEA real index and its growth rate, the bankng sector non-performing loan (NPL ratio), and the capital adequacy ratio.. I also use BoG's series on export, imports, trade balance, gross and net reserves, and the interbank exchange rate
of their currency with USD. I also collect dates for the Ghana IMF arrangements from the IMF lending database.

All series are merged into monthly data set with a DateTime index. For most nominal stock variables such as M2+ and private credit I work with year-on-year percentage changes. I also adjust private credit year-over-year for inflation to get real credit.
For the policy rate and other interest rates I compute the real versions by subtracting the inflation. For the exchange rate I calculate a 12-month depreciation rate which measures how much the cedi lost against the dollar over the previous year. For reserves
I compute the 12-month change in net reserves.

First I make time-series and scatter plots to see whether the variables move together. Second, I calculate correlations and run a few on-variable OLS regressions. For IMF episodes, I mark the start dates of each on top of the exchange rate depreciation series
and compare the averge depreciation in a window around IMF programs.


\section{Results}

\subsection{Inflation vs Policy Rate}

\begin{figure}[htbp]
  \centering
  \includegraphics[width=0.8\textwidth]{../figures/analysis/inflationPolicyRateRealPolicyRate.png}
  \caption{Time series - headline inflation, policy rate, real policy rate.}
  \label{fig:inflationvspolicy}
\end{figure}

Figure 1 shows the headline inflation together with the BoGs policy rate and ex-post real policy from 2007 to 2024. Three big inflation periods stand out being around 2008-09, mid-2014 to 2016 and the recent covid spike around 2022. In each of these
instances, the policy rate eventually moves up, but the real policy rate becomes sharply negative when inflation jumps faster than the nominal rate. 

This suggests that Ghanaian monetary policy is often behind the curve in the short run. In an AD-AS or IS-LM model, high inflation with negative real rates correspond with aggregate demand being above potential output. Demand is being supported even while prices
are rising rapidly. Only once the policy rate catches and the real rate return toward zero does the inflation fall back.

\begin{figure}[htbp]
  \centering
  \includegraphics[width=0.8\textwidth]{../figures/analysis/MoneyGrowthVsInflationTimeSeries.png}
  \caption{Time series - money growth and inflation headline}
  \label{fig:moneygrowthandinflation}
\end{figure}

\begin{figure}[htbp]
  \centering
  \includegraphics[width=0.8\textwidth]{../figures/analysis/MoneyGrowthVsInflationScatter.png}
  \caption{Scatter - money growth vs inflation headline}
  \label{fig:moneygrowthvsinflation}
\end{figure}

To link this to money growth, Figure 2 and 3 show headline inflation against broad money year-on-year growth. In Figure 2, the two series are not perfectly aligned, but they do tend to spike together in the three main crisis periods. 
Running an OLS regression on inflation on M2+ growth gives a positive and statistically significant coefficient of about 0.30, indicating that, on average, a 10-percentage point increase in money growth is associated with a roughly 3-percentage point increase
in inflation. However, the R-Squared is only about 0.08 which means that money growth by itself explains less than 10 percent of the month-to-month variation in inflation. This matches the idea that, in the short run, many other factors matter for inflation such 
as supply shocks, fiscal policy, and expectations.

I also look at how the central bank reacts to inflation by regressing monthly changes in the policy rate on lagged inflation. The estimated coefficient is positive and significant, showing that, when inflation is 10 pp higher, the policy rate tends to be raised by about
0.4 pp in the next month. The R-squared is barely above 0.1 which shows that the BoG responds to inflation but only gradually.

\subsection{Credit, Growth, and Bank}

\begin{figure}[htbp]
  \centering
  \includegraphics[width=0.8\textwidth]{../figures/analysis/RealCreditGrowthVsRealActivity.png}
  \caption{Time series - private credit and real growth}
  \label{fig:privatecreditandrealgrowth}
\end{figure}

Figure 4 shows real private credit growth with the BoG's CIEA real growth rate. Real credit growth is very volatile, being super high in the late 2000s and slowing sharply in 2012, collapsing again in 2017 and 2020. The CIEA growth rate is much
smoother but clearly weakens around the same time periods. Periods of strong credit expansion tend to coincide with stronger real activity.

A regression of CIEA growth on real credit growth gives a coefficient of about 0.40 that is statistically significant, with an R-squared around 0.19. This means that when real private credit growth is 10 pp higher the CIEA growth rate is on average 4 p higher
the same year. This relationship is not great but strong enough to show the importance of bank lending in Ghana's business cycle.

When I shift the real credit growth three months back and regress CIEA growth on the lagged credit, the coefficient stays poitive but smaller and the R-squared drops to about 0.06. This suggests that credit booms may lead real activity with slight lag. Using IS-LM,
this matches the idea that more bank lending shifts the IS crive to the right and temporarily increases output and incomes.

\begin{figure}[htbp]
  \centering
  \includegraphics[width=0.8\textwidth]{../figures/analysis/CreditGrowthVsNPLratioScatter.png}
  \caption{Scatter - npl ratio vs private credit yoy}
  \label{fig:nplvscredit}
\end{figure}

Bank health is also very important. Figure 5 shows the NPL ratio against nominal private credit growth. Here the relationship is negative with the credit growth tending to be low when NPL is high, and vice versa. This is very consistent with a cycle where we
see, after a lending boom, bad loans piling up and banks beginning to cut back on new lending. For Ghana, this is very visible around 2016-2018 when NPLs were elevated and credit growth was a lot slower. This all makes sense because a high NPL ratio would
typically tighten financial conditions, pushing loanable funds to the left and reducing investment and growth.

\subsection{Exchange Rate, Reserves, and IMF Programs}

\begin{figure}[htbp]
  \centering
  \includegraphics[width=0.8\textwidth]{../figures/analysis/FXvsReservesDualAxis.png}
  \caption{Time Series - dual axis mid price and net reserves}
  \label{fig:midandreserves}
\end{figure}

\begin{figure}[htbp]
  \centering
  \includegraphics[width=0.8\textwidth]{../figures/analysis/FXDepreciationVsReserveChangesScatter.png}
  \caption{Scatter - fx depreciation vs reserve changes}
  \label{fig:fxvsreserve}
\end{figure}

On the external side, Figure 6 shows the interbank exchange rate (GHS per USD) together with Ghana's net international reserves. The cedi depreciates gradually in the late 2000s, more sharply around 2014-15, and again very rapidly in 2022. Reserves rise in
the early 2010s but fall again around the time of the big depreciations. This shows that the central bank is attempting to smooth the exchange rate by selling foreign currency.

The scatter plot of the 12-month FX depreciation against the 12-month change in net reserves makes this relationship much more systematic. The regression slope is strongly negative and highly significant with an R-squared of about 0.36. Years in which 
net reserves increase tend to have much lower exchange rate depreciation, while years when reserves were run down tend to coincide with big loss in cedi value relative to USD. This is exactly what is to be expected from an AD-AS model. Using reserves to
smooth the exchange rate can temporarily contain a depreciation, but when reserves are falling super quickly it is usually a sign that currency will end up weakening.

\begin{figure}[htbp]
  \centering
  \includegraphics[width=0.8\textwidth]{../figures/analysis/FXDepreciationAndIMFPrograms.png}
  \caption{Time Series - fx depreciation with IMF program start dates}
  \label{fig:fxandimf}
\end{figure}

To connect this to IMF support, Figure 8 overlays the vertical lines for the start of each IMF arrangement on top of the FX depreciation series. The pattern is very telling. Large spikes in depreciation are typically clustering around the start dates of new IMF
programs, although we are limited to three data points. When I compare average 12-month depreciation inside an 18-month window around IMF program start dates to the rest of the sample, the IMF window periods show much higher average depreciation (around forty percent)
than normal times (about 16 percent). This may indicate that Ghana tends to turn to the IMF only once a balance-of-payment crisis is already fairly severe, instead of pre-emptively.

Fiscal and monetary expansion, along with external shocks, push the demand above potential and widen external imbalances, putting more pressure on the currency. When reserves and investor confidence are low, the country ends up at the IMF for financing and policy
help. The IMF program is a sort of point where policy hgas to shift back toward tigher fiscal and monetary stances to stabilize the currency and inflation.

\section{Policy Recommendations}

Putting these pieces together, a few policy ideas emerge. First, inflation in Ghana is clearly related to money and credit growth, but the central bank's policy rate often adjusts too slowly. During the major inflation episodes the real policy rate
becomes strongly negative, which is equivalent to continued stimulus even when the economy overheats. A more systematic response would be to raise the policy rate more aggressively when inflation deviates from the target. This could help to limit how far inflation
is allowed to run before being brought back down. 

Second, the link between real credit growth and real activity suggests that tightening policy too much or allowing the banking system to stay weak can have major impacts on growth and employment. Real credit growth is correlated with the CIEA's real growth rate,
and high NPL ratios are associated with lower credit growth. This points to the importance of strong bank supervision and being able to do recapitalization at good times, so that credit can expand in a way that avoids creating new waves of bad loans.

Third, the external side of the story shows how vulnerable Ghana is to swings in capital flows and commodity prices. Exchange rate depreciation is very closely tied to changes in international reserves and large depreciations tend to occur right before IMF programs.
In good years, Ghana needs to build up reserves and maintain credible fiscal and monetary frameworks so that it has buffers when conditiions become bad. If reserves are already low when policy is loose and debt is high, the country has minimal room to avoid having to
make a painful adjustment later.

These conclusions match those of the AD-AS and IS-LM diagrams. Expansionary fiscal policy and monetary policy can shift aggregate demand to the right and raise output in the short run, but if not reversed in time, can cause the economy to have higher inflation, 
currency depreciation, and lower growth overall. The challenge for Ghanaian policymakers is to use these tools to lean against the booms and not just reactively when crises is hit. 

\section{Conclusion}



\bibliographystyle{apalike}      
\bibliography{references}        

\end{document}


